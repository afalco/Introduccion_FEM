\documentclass[a4paper,10pt]{article}
%%%%%%%%%%%%%%%%%%%%%%%%%%%%%%%%%%%%%%%%%%%%%%%%%%%%%%%%%%%%%%%%%%%%%%%%%%%%%%%%%%%%%%%%%%%%%%%%%%%%%%%%%%%%%%%%%%%%%%%%%%%%%%%%%%%%%%%%%%%%%%%%%%%%%%%%%%%%%%%%%%%%%%%%%%%%%%%%%%%%%%%%%%%%%%%%%%%%%%%%%%%%%%%%%%%%%%%%%%%%%%%%%%%%%%%%%%%%%%%%%%%%%%%%%%%%
\usepackage{eurosym}
\usepackage{makeidx}
\usepackage{amsfonts}
\usepackage{amsmath}
%\usepackage[UKenglish]{babel}
\usepackage{graphicx}
\usepackage{amssymb}
\usepackage{mathrsfs}
\usepackage{graphicx}
\usepackage[all,cmtip]{xy}
\usepackage{synttree}
\usepackage{tikz}
\usepackage{color}

\setcounter{MaxMatrixCols}{10}


\providecommand{\U}[1]{\protect\rule{.1in}{.1in}}
\newtheorem{theorem}{Theorem}[section]
\newtheorem{algorithm}[theorem]{Algorithm}
\newtheorem{assumption}[theorem]{Assumption}
\newtheorem{condition}[theorem]{Condition}
\newtheorem{conjecture}[theorem]{Conjecture}
\newtheorem{construction}[theorem]{Construction}
\newtheorem{conclusion}[theorem]{Conclusion}
\newtheorem{corollary}[theorem]{Corollary}
\newtheorem{criterion}[theorem]{Criterion}
\newtheorem{definition}[theorem]{Definition}
\newtheorem{example}[theorem]{Example}
\newtheorem{lemma}[theorem]{Lemma}
\newtheorem{notation}[theorem]{Notation}
\newtheorem{proposition}[theorem]{Proposition}
\newtheorem{problem}[theorem]{Problem}
\newtheorem{remark}[theorem]{Remark}
\newtheorem{summary}[theorem]{Summary}
\newenvironment{proof}[1][Proof]{\noindent \emph{#1.} }{\hfill \ \rule{0.5em}{0.5em}}
\newcommand{\BIGOP}[1]{\mathop{\mathchoice{\raise-0.22em\hbox{\huge $#1$}} {\raise-0.05em\hbox{\Large $#1$}}{\hbox{\large $#1$}}{#1}}}
\renewcommand{\theequation}{\thesection.\arabic{equation}}
\renewcommand{\thefigure}{\thesection.\arabic{figure}}
\renewcommand{\thetable}{\thesection.\arabic{table}}
\makeatletter\@addtoreset{equation}{section}\makeatother
\makeatletter\@addtoreset{figure}{section}\makeatother
\makeatletter\@addtoreset{table}{section}\makeatother
 \textheight23cm \textwidth16.5cm
 \oddsidemargin-.0cm
 \evensidemargin-.32cm
\newcommand{\red}[1]{{\textcolor{red}{#1}}}
%\input{tcilatex}

\newcommand{\comment}[1]{{\color{red}{#1}}}

\begin{document}

\title{Métodos numéricos en Ingeniería: Ecuaciones diferenciales y en derivadas parciales (Método de los Elementos Finitos)}
\author{F. Chinesta y A. Falcó}
\date{A entregar el 21 de Enero de 2022}
\maketitle

\begin{enumerate}
\item Consideremos la función de $H_0^2([0,1])$ definida por
$$
u(x)= \left\{
\begin{array}{cc}
-\frac{x^2}{2}+3x & \text{ si } 0 \le x < 1/3, \\ 
2 & \text{ si } x=1/3, \\ 
2x +1 & \text{ si } 1/3 < x < 2/3, \\ 
2 & \text{ si } x= 2/3 \\ 
\frac{x^2}{2}+5x & \text{ si } 2/3 < x < 1.
\end{array}
\right.
$$
Sea $f \in L_2([0,1])$ la función definida por
$$
f(x)= \left\{
\begin{array}{cc}
-\frac{x^2}{2}+3x +1 & \text{ si } 0 \le x < 1/3, \\ 
2x +1 & \text{ si } 1/3 \le  x \le   2/3, \\ 
\frac{x^2}{2}+5x -1& \text{ si } 2/3 \le  x < 1.
\end{array}
\right.
$$
Demuestra que $u$ es solución débil del problema a valor frontera 
\begin{align}
- u^{\prime \prime}(x) + u(x) & = f(x) \text{ en } 0 < x < 1 \label{ODE1} \\ 
 u(0) = u(1) = 0.  \label{ODE2}
\end{align}
\item Calcula empleando el método de los elementos finitos una aproximación 
$\widehat{u} \in \mathbb{R}^{d-2}$ del problema a valor frontera \eqref{ODE1}-\eqref{ODE2}, con
$d=50,100,150,200,250,300,350,400,450,500$ y efectúa un análisis del error de aproximación, empleando
los valores entre la solución exacta y la solución aproximada.
\item Calcula empleando un método de diferencias finitas (implícito o explícito) una aproximación 
$\tilde{u} \in \mathbb{R}^{d-2}$ del problema a valor frontera \eqref{ODE1}-\eqref{ODE2}, con
$d=50,100,150,200,250,300,350,400,450,500$ y efectúa un análisis del error de aproximación, empleando
los valores entre la solución exacta y la solución aproximada.
\item Compara las soluciones aproximadas $\widehat{u}$ obtenidas en el apartado 2 y las soluciones aproximadas 
$\tilde{u}$ obtenidas en el apartado 3.
\item ¿Qué conclusiones podemos extraer de la comparación efectuada en el apartado anterior? 
\end{enumerate}




\end{document}

\begin{thebibliography}{99}

\bibitem{AmmarChinestaFalco2010} A. Ammar, F. Chinesta and A. Falc\'o, On the convergence of a Greedy Rank-One Update Algorithm for a class of Linear Systems . \emph{Archives of Computational Methods in Engineering}, Volume 17, Number 4, (2010), 473-486.

\bibitem{FH} A. Falcó and W. Hackbusch. On minimal subspaces in tensor representations. \emph{Foundations of Computational Mathematics}, Volume 12, Issue 6 (2012), pp 765-803 .

\end{thebibliography}
